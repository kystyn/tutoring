\section{Логические операции}

\begin{definition}
	Будем говорить, что операция является \textbf{сильной относительно значения}, если она принимает это значение \textbf{реже}, чем противоположное.
\end{definition}

\begin{definition}
	Будем говорить, что операция является \textbf{слабой относительно значения}, если она принимает это значение \textbf{чаще}, чем противоположное.
\end{definition}
Пример: операция конъюнкции является сильной относительно единицы, поскольку она принимает её лишь в одном случае из четырёх возможных.

\subsection{Логическое И}
\subsubsection{Таблица истинности}
\begin{table}[h]
	\begin{center}
		\begin{tabular}{|c|c|c|}
			\hline
			$x$ & $y$ & $x\&y$\\
			\hline
			0 & 0 & 0\\
			\hline
			0 & 1 & 0\\
			\hline
			1 & 0 & 0\\
			\hline
			1 & 1 & 1\\
			\hline
		\end{tabular}
		\caption{Логическое И}
	\end{center}
\end{table}

\subsubsection{Обозначения}
Операция также называется \textbf{конъюнкцией}, обозначается через $\bigwedge$, AND, И.
\begin{remark}
	Не путать в C-подобных языках (и Java) с обозначением исключающего или!
\end{remark}

\subsubsection{Свойства}
\begin{enumerate}
	\item Данная операция полностью эквивалентна обыкновенному умножению, что легко проверяется подстановкой.
	\item Данная операция является слабой относительно нуля.
\end{enumerate}

\subsection{Логическое ИЛИ}
\subsubsection{Таблица истинности}
\begin{table}[h]
	\begin{center}
		\begin{tabular}{|c|c|c|}
			\hline
			$x$ & $y$ & $x\bigvee y$\\
			\hline
			0 & 0 & 0\\
			\hline
			0 & 1 & 1\\
			\hline
			1 & 0 & 1\\
			\hline
			1 & 1 & 1\\
			\hline
		\end{tabular}
		\caption{Логическое ИЛИ}
	\end{center}
\end{table}

\subsubsection{Обозначения}
Операция также называется \textbf{дизъюнкцией}, обозначается через $\textbar$, OR, ИЛИ.

\subsubsection{Свойства}
\begin{enumerate}
	\item Данная операция является слабой относительно единицы.
\end{enumerate}

\subsection{Отрицание}
\subsubsection{Таблица истинности}
\begin{table}[h]
	\begin{center}
		\begin{tabular}{|c|c|}
			\hline
			$x$ & $\neg x$\\
			\hline
			0 & 1\\
			\hline
			1 & 0\\
			\hline
		\end{tabular}
		\caption{Отрицание}
	\end{center}
\end{table}

\subsubsection{Обозначения}
Операция также называется \textbf{инверсией}. Имеет место обозначение $\overline{x}$.

\subsubsection{Свойства}
\begin{enumerate}
	\item Данная операция не является сильной или слабой.
	\item Операция полностью эквивалентна арифметическому действию $\neg{x} := 1 - x$, что легко проверяется подстановкой.
\end{enumerate}


\subsection{Импликация}
\subsubsection{Таблица истинности}
\begin{table}[h]
	\begin{center}
		\begin{tabular}{|c|c|c|}
			\hline
			$x$ & $y$ & $x\rightarrow y$\\
			\hline
			0 & 0 & 1\\
			\hline
			0 & 1 & 1\\
			\hline
			1 & 0 & 0\\
			\hline
			1 & 1 & 1\\
			\hline
		\end{tabular}
		\caption{Импликация}
	\end{center}
\end{table}

\subsubsection{Обозначения}
Операция также называется \textbf{следованием}.

\subsubsection{Свойства}
\begin{enumerate}
	\item Данная операция является слабой относительно единицы.
	\item Операция полностью эквивалентна $\neg x \bigvee y$.
\end{enumerate}
\subsection{Эквиваленция}
\subsubsection{Таблица истинности}
\begin{table}[h]
	\begin{center}
		\begin{tabular}{|c|c|c|}
			\hline
			$x$ & $y$ & $x\leftrightarrow y$\\
			\hline
			0 & 0 & 0\\
			\hline
			0 & 1 & 0\\
			\hline
			1 & 0 & 0\\
			\hline
			1 & 1 & 1\\
			\hline
		\end{tabular}
		\caption{Эквиваленция}
	\end{center}
\end{table}

\subsubsection{Обозначения}
Операция также называется \textbf{равносильностью}.

\subsubsection{Свойства}
\begin{enumerate}
	\item Данная операция не является сильной.
	\item Операция полностью эквивалентна $x \rightarrow y \& y \rightarrow x$.
\end{enumerate}

\subsection{Исключающее или}
\subsubsection{Таблица истинности}
\begin{table}[h]
	\begin{center}
		\begin{tabular}{|c|c|c|}
			\hline
			$x$ & $y$ & $x +_2 y$\\
			\hline
			0 & 0 & 0\\
			\hline
			0 & 1 & 1\\
			\hline
			1 & 0 & 1\\
			\hline
			1 & 1 & 0\\
			\hline
		\end{tabular}
		\caption{Исключающее или}
	\end{center}
\end{table}

\subsubsection{Обозначения}
Имеют место обозначения: $\bigwedge$ в С-подобных языках и Java, XOR.

\subsubsection{Свойства}
\begin{enumerate}
	\item Операция также называется \textbf{сложением по модулю 2}, поскольку является семантически эквивалентной ей: $x +_2 y := (x + y) mod 2$.
	\item Данная операция не является сильной.
	\item Операция является обратной к эквиваленции.
\end{enumerate}


