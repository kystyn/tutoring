\section{Логические операции}

\begin{definition}
	Будем говорить, что операция является \textbf{сильной относительно значения}, если она принимает это значение \textbf{реже}, чем противоположное.
\end{definition}

\begin{definition}
	Будем говорить, что операция является \textbf{слабой относительно значения}, если она принимает это значение \textbf{чаще}, чем противоположное.
\end{definition}
Пример: операция конъюнкции является сильной относительно единицы, поскольку она принимает её лишь в одном случае из четырёх возможных.

\subsection{Логическое И}
\subsubsection{Таблица истинности}
\begin{table}[H]
	\begin{center}
		\begin{tabular}{|c|c|c|}
			\hline
			$x$ & $y$ & $x\&y$\\
			\hline
			0 & 0 & 0\\
			\hline
			0 & 1 & 0\\
			\hline
			1 & 0 & 0\\
			\hline
			1 & 1 & 1\\
			\hline
		\end{tabular}
		\caption{Логическое И}
	\end{center}
\end{table}

\subsubsection{Обозначения}
Операция также называется \textbf{конъюнкцией}, обозначается через $\bigwedge$, AND, И. В C-подобных языыках и Java обозначается $\&$.
\begin{remark}
	Не путать с обозначением \textbf{исключающего ИЛИ} (\ref{xor}) в C-подобных языках (и Java)!
\end{remark}


\subsubsection{Свойства}
\begin{enumerate}
	\item Данная операция полностью эквивалентна обыкновенному умножению, что легко проверяется подстановкой.
	\item Данная операция является слабой относительно нуля.
\end{enumerate}

\subsection{Логическое ИЛИ}
\subsubsection{Таблица истинности}
\begin{table}[H]
	\begin{center}
		\begin{tabular}{|c|c|c|}
			\hline
			$x$ & $y$ & $x\bigvee y$\\
			\hline
			0 & 0 & 0\\
			\hline
			0 & 1 & 1\\
			\hline
			1 & 0 & 1\\
			\hline
			1 & 1 & 1\\
			\hline
		\end{tabular}
		\caption{Логическое ИЛИ}
	\end{center}
\end{table}

\subsubsection{Обозначения}
Операция также называется \textbf{дизъюнкцией}, обозначается через $\textbar$, OR, ИЛИ. В C-подобных языыках и Java обозначается $|$.

\subsubsection{Свойства}
\begin{enumerate}
	\item Данная операция является слабой относительно единицы.
\end{enumerate}

\subsection{Отрицание}
\subsubsection{Таблица истинности}
\begin{table}[h]
	\begin{center}
		\begin{tabular}{|c|c|}
			\hline
			$x$ & $\neg x$\\
			\hline
			0 & 1\\
			\hline
			1 & 0\\
			\hline
		\end{tabular}
		\caption{Отрицание}
	\end{center}
\end{table}

\subsubsection{Обозначения}
Операция также называется \textbf{инверсией}. Имеет место обозначение $\overline{x}$. В C-подобных языыках и Java обозначается $\sim$.

\subsubsection{Свойства}
\begin{enumerate}
	\item Данная операция не является сильной или слабой.
	\item Операция полностью эквивалентна арифметическому действию $\neg{x} := 1 - x$, что легко проверяется подстановкой.
\end{enumerate}


\subsection{Импликация}
\subsubsection{Таблица истинности}
\begin{table}[h]
	\begin{center}
		\begin{tabular}{|c|c|c|}
			\hline
			$x$ & $y$ & $x\rightarrow y$\\
			\hline
			0 & 0 & 1\\
			\hline
			0 & 1 & 1\\
			\hline
			1 & 0 & 0\\
			\hline
			1 & 1 & 1\\
			\hline
		\end{tabular}
		\caption{Импликация}
	\end{center}
\end{table}

\subsubsection{Обозначения}
Операция также называется \textbf{следованием}.

\subsubsection{Свойства}
\begin{enumerate}
	\item Данная операция является слабой относительно единицы.
	\item Операция полностью эквивалентна $\neg x \bigvee y$.
\end{enumerate}
\subsection{Эквиваленция}
\subsubsection{Таблица истинности}
\begin{table}[H]
	\begin{center}
		\begin{tabular}{|c|c|c|}
			\hline
			$x$ & $y$ & $x\leftrightarrow y$\\
			\hline
			0 & 0 & 1\\
			\hline
			0 & 1 & 0\\
			\hline
			1 & 0 & 0\\
			\hline
			1 & 1 & 1\\
			\hline
		\end{tabular}
		\caption{Эквиваленция}
	\end{center}
\end{table}

\subsubsection{Обозначения}
Операция также называется \textbf{равносильностью}, \textbf{тождественным равенством}. В ЕГЭ имеет место обозначение $\equiv$.

\subsubsection{Свойства}
\begin{enumerate}
	\item Данная операция не является сильной.
	\item Операция полностью эквивалентна $x \rightarrow y \& y \rightarrow x$.
	\item Именование ``тождественное равенство'' дано неслучайно: в самом деле, операция эквивалентна обыкновенному равенству
	
\end{enumerate}

\subsection{Исключающее ИЛИ} \label{xor}
\subsubsection{Таблица истинности}
\begin{table}[H]
	\begin{center}
		\begin{tabular}{|c|c|c|}
			\hline
			$x$ & $y$ & $x +_2 y$\\
			\hline
			0 & 0 & 0\\
			\hline
			0 & 1 & 1\\
			\hline
			1 & 0 & 1\\
			\hline
			1 & 1 & 0\\
			\hline
		\end{tabular}
		\caption{Исключающее ИЛИ}
	\end{center}
\end{table}

\subsubsection{Обозначения}
Обозначается XOR (eXclusive OR).
В C-подобных языыках и Java обозначается $\bigwedge$.

\subsubsection{Свойства}
\begin{enumerate}
	\item Операция также называется \textbf{сложением по модулю 2}, поскольку является семантически эквивалентной ей: $x +_2 y := (x + y) mod 2$.
	\item Данная операция не является сильной.
	\item Операция является обратной к эквиваленции.
\end{enumerate}

\subsection{Практическое применение}
\begin{remark}
	Языки программирования различают битовые и логические операции И, ИЛИ, НЕ, чем вносят некоторую путаницу. 
	
	\begin{table}[H]
		\begin{center}
			\begin{tabular}{|c|c|c|}
				\hline
				Название & Логическая & Битовая\\
				\hline
				И & \&\& & \&\\
				\hline
				ИЛИ & || & |\\
				\hline
				НЕ & $\sim$ & !\\
				\hline
			\end{tabular}
			\caption{Некоторые операции в языках программирования}
		\end{center}
	\end{table}
	
	Так, с точки зрения языка Си, аргументы \textbf{логического И} обрабатываются следующим образом: если аргумент не равен нулю, он воспринимается как истинностное значение. Ноль -- как ложное. \textbf{Логического И} всегда возвращает \textbf{логический} результат -- истину или ложь (в языке Си это 1 или 0, по умолчанию тип int, в Java -- true или false, тип boolean).
	
	 \textbf{Битовое И}, однако, ведёт себя абсолютно иным образом: команда представляет аргументы в двоичном виде, делает поразрядную (побитовую) конъюнкцию и полученный результат переводит обратно в десятичный вид. Конечно же, это реализуется иным образом, иначе эффективность операций была неоправданно низкой, однако человек бы делал то, что делает машина, именно таким образом (по крайней мере, наивную реализацию). \textbf{Битовое И} всегда возвращает \textbf{целочисленный} результат -- целое число из разрешённого диапазона.
	
	Пример: 
	\begin{itemize}
		\item 123 \&\& 46 == ИСТИНА И ИСТИНА == ИСТИНА.
		\item 123 \& 46 == $1111011_2$ И $0101110_2$ == $0101010_2$ == 41
	\end{itemize}
	
	По аналогичному принципу работают остальные битовые операции.
	
	Итого: \textbf{битовые} операции переводят целое число в двоичный формат и производят поразрядное применение \textbf{логической} операции в соответствии с построенными таблицами истинности.
\end{remark}

\begin{remark}
	Всё вышесказанное в полной мере относится и к Java.
\end{remark}

\begin{remark}
	Операция исключающего ИЛИ -- только \textbf{битовая}! Операция имликации и иже с ними не реализована в языке, поскольку является выразимой через уже реализованные.
\end{remark}

\begin{remark} \label{cpu-bit}
	Битовые операции поддерживаются подавляющим большинством современных процессоров на аппаратном уровне, поэтому их выполнение осуществляется очень быстро. В связи с этим многие операции, родственные битовым, такие как генерация подмножеств заданного множества, стараются осуществлять с помощью так называемых битовых хаков.
\end{remark}
