\section{Задания на логические операции}
\subsection{Задание №2}
На 2020 год задание формулируется следующим образом: дана некоторая логическая функция, обычно именуемая $F$, её аргументы -- $x, y, z, w$. Обычно число аргументов колеблется в пределах 3--4. Для данной функции приведён фрагмент таблицы истинности, причём функция на приведённом в условии множестве аргументов принимает только одно логическое значение, то есть не встречается условий, в которых в одной части строк функция обращается в ложь, а в другой -- в истину. Это замечание является важным для приведённого ниже алгоритма решения. 

В строке таблицы истинности указываются не все значения, но лишь некоторые. Не встречается пустых строк, в которых указано только значение функции, но не указаны значения её аргументов.


Требуется выяснить, в каком порядке указаны аргументы во фрагменте таблицы истинности.

\textbf{Алгоритм действия} в данном задании следующий:
\begin{enumerate}
	\item Первый этап.
	\begin{itemize}
		\item Посмотреть в условие задачи. Выяснить, для какого значения -- истина или ложь -- приводится часть таблицы, задающей логическую функцию.
		\item Построить (мысленно или на бумаге -- как удобно) дерево операций по следующему принципу: в узле дерева находится операция, на левой и правой ветви - её аргументы.
		\item Выполняя обход в глубину, проанализировать для каждого уровня дерева, является ли операция, на нём стоящая, слабой или сильной относительно того значения, для которого приводится фрагмент таблицы истинности функции.
	\end{itemize}
	\item Второй этап.
	\begin{itemize}
		\item В первую очередь следует выписать комбинации значений для сильных операций, поскольку они задают наиболее жёсткие условия. Далее -- по мере ослабления операции.
		\item На данный момент должна быть сформирована таблица истинности для приведённого в условии значения функции.
	\end{itemize}
	\item Третий этап.
	\begin{itemize}
		\item Проанализировать построенную таблицу истинности: наверняка будет ситуация, когда один или несколько столбцов и/или строк полностью заполнен(а) одним значением.
		\item В приведённой в задании таблице следует отбраковать все строки и столбцы, не подходящие для найденных из предыдущих пунктов. Как правило, на данном этапе окажется найденной позиция одного или даже двух аргументов.
		\item Оставшиеся столбцы заполняются, если возможно, методом исключения (остался незаполненным один столбец и один аргумент соответственно), либо перебором. 
		\item Задание выполнено!
	\end{itemize}
\end{enumerate}